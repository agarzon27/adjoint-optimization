% ****** Start of file apssamp.tex ******
%
%   This file is part of the APS files in the REVTeX 4.2 distribution.
%   Version 4.2a of REVTeX, December 2014
% 
%   Copyright (c) 2014 The American Physical Society.
%
%   See the REVTeX 4 README file for restrictions and more information.
%
% TeX'ing this file requires that you have AMS-LaTeX 2.0 installed
% as well as the rest of the prerequisites for REVTeX 4.2
%
% See the REVTeX 4 README file
% It also requires running BibTeX. The commands are as follows:
%
%  1)  latex apssamp.tex
%  2)  bibtex apssamp
%  3)  latex apssamp.tex
%  4)  latex apssamp.tex
%
% \documentclass[%
% reprint,
% %superscriptaddress,
% %groupedaddress,
% %unsortedaddress,
% %runinaddress,
% %frontmatterverbose, 
% %preprint,
% %preprintnumbers,
% %nofootinbib,
% %nobibnotes,
% %bibnotes,
%  amsmath,amssymb,
%  aps,
% %pra,
% %prb,
% %rmp,
% %prstab,
% %prstper,
% %floatfix,
% ]{revtex4-2}
% \documentclass[twocolumn]{article}
\documentclass{article}
\synctex=-1

\usepackage{amsmath,amssymb}
\usepackage{url}
\usepackage{graphicx}% Include figure files
% \usepackage{dcolumn}% Align table columns on decimal point
\usepackage{bm}% bold math
% \usepackage{diagbox}
% \usepackage{booktabs}
% \usepackage{tabularx}
% \usepackage[ruled,linesnumbered]{algorithm2e}
% %\usepackage{algpseudocode}
% %\usepackage{algorithmic}
% \usepackage{dblfloatfix}
% \usepackage{multirow}
% \usepackage{threeparttable}
\usepackage[colorlinks]{hyperref}
\hypersetup{
    colorlinks = true,
    linkcolor=blue
  }
% \usepackage[table]{xcolor}
% \usepackage{makecell}
  
%\usepackage{siunitx}
%\usepackage{hyperref}% add hypertext capabilities
%\usepackage[mathlines]{lineno}% Enable numbering of text and display math
%\linenumbers\relax % Commence numbering lines

%\usepackage[showframe,%Uncomment any one of the following lines to test 
%%scale=0.7, marginratio={1:1, 2:3}, ignoreall,% default settings
%%text={7in,10in},centering,
%%margin=1.5in,
%%total={6.5in,8.75in}, top=1.2in, left=0.9in, includefoot,
%%height=10in,a5paper,hmargin={3cm,0.8in},
%]{geometry}

\newcommand{\tp}{\mathsf{T}}
% \newcommand{\wt}[1]{\widetilde{#1}}
% \newcommand{\wh}[1]{\widehat{#1}}
% \newcolumntype{Y}{>{\centering\arraybackslash}X}
% \newcolumntype{C}{>{\centering\arraybackslash}p{1.6cm}}
% \newcolumntype{D}{>{\centering\arraybackslash}p{1cm}}
% \newcolumntype{E}{>{\centering\arraybackslash}p{3cm}}

% \newcolumntype{F}[1]{>{\centering\arraybackslash}p{#1}}
\usepackage[left=2cm,right=2cm,top=2cm,bottom=2cm]{geometry}
\usepackage[utf8]{inputenc}

\usepackage{color,listings}
\lstset{breaklines=true,breakindent=0pt,
  prebreak=\mbox{\tiny$\searrow$},
  postbreak=\mbox{{\color{blue}\tiny$\rightarrow$}}}

% \newcommand{\code}[1]{{\small\texttt{#1}}}

\definecolor{darkgray}{rgb}{0.66, 0.66, 0.66}
\definecolor{darkorange}{rgb}{1.0, 0.55, 0.0}
% \definecolor{gray}{rgb}{0.97,0.97,0.99}
\definecolor{gray}{rgb}{0.9,0.9,0.9}
\definecolor{teal}{rgb}{0.0, 0.5, 0.5}
\definecolor{comment}{rgb}{0.6, 0, 0.9}
\lstdefinestyle{mystyle}{
%	language = Python,
	backgroundcolor=\color{gray},
	commentstyle=\color{comment},
	keywordstyle=\bfseries\color{darkorange},
	numberstyle=\scriptsize\color{darkgray},
	stringstyle=\color{teal},
	% basicstyle=\scriptsize\ttfamily,%\linespread{1}
	basicstyle=\ttfamily,%\linespread{1}
	breakatwhitespace=false,
	breaklines=true,
	captionpos=t,
	keepspaces=false,
	% numbers=left,
	numbersep=3pt,
	showspaces=false,
	showstringspaces=false,
	showtabs=false,
	tabsize=4
}
\lstset{style=mystyle}

\newcommand{\code}[1]{\colorbox{gray}{\lstinline|#1|}}

\begin{document}

%\preprint{APS/123-QED}

\title{A set of MATLAB-CUDA programs to find low-energy defibrillating signals by adjoint optimization}
\author{First Author}
 % \altaffiliation[Also at ]{Physics Department, XYZ University.}%Lines break automatically or can be forced with \\
\author{Alejandro Garz\'on}%
%  \email{Second.Author@institution.edu}
% \affiliation{%
%  Authors' institution and/or address
% }%

\date{\today}% It is always \today, today,
             %  but any date may be explicitly specified

%\keywords{Suggested keywords}%Use showkeys class option if keyword
                              %display desired
\maketitle
\begin{abstract}
  This document explains the compilation and operation of a set of MATLAB-CUDA programs for finding low-energy defibrillating signals in electrophysiological models. The programs implement the procedure explained in Refs. \cite{garzon2024ultra,garzon2024chaos}, based on the minimization of a functional of the signal, defined over the model solution. The optimization is performed by the gradient descent approach, with the functional gradient computed efficiently by the adjoint method.
\end{abstract}


% \tableofcontents
\section{Method description}
References \cite{garzon2024ultra,garzon2024chaos} present a method for finding low-energy electric field signals $E(t)$ capable of producing defibrillation in electrophysiological models of two-dimensional cardiac tissue. The method is based on the minimization of a functional $\mathcal{L}$ of $E(t)$,
\begin{equation}
  \label{eq:LNM}
      \mathcal{L} = \frac{1}{2}\mathcal{M} + \frac{\alpha}{2}\mathcal{N},
\end{equation}
that penalizes the energy $\mathcal{N}$ of $E(t)$ and the spatial variation $\mathcal{M}$ of the model state variables at a given time $T$. $\mathcal{L}[E(t)]$ is minimized by the gradient descent approach with the functional gradient $\mathcal{G}(t)$ computed efficiently by the adjoint method. To explain the installation and operation of the software that computes the optimal $E(t)$, a sketchy presentation of the method equations follows (the full details can be found in Ref. \cite{garzon2024ultra}).

After space discretization, the model's partial differential equations become a system of ordinary differential equations (ODEs), with a huge number of state variables, in the order of tens of thousands. If the state variables are gathered in a vector ${\bf w}$, the ODE system takes the form
\begin{equation}
  \label{eq:1}
  \dot{\bf w} = L{\bf w} + F({\bf w}) + E(t){\bf b}.  
\end{equation}
% \begin{subequations}
% \label{eq:main_ivp_bfu}
% \begin{align}
%   \label{eq:main_dot_bfu}
%   \dot{\bf w} &= L{\bf w} + F({\bf w}) + E{\bf b},\\
%   \label{eq:main_dot_bfuT}
%   {\bf w}(0) &= {\bf w}_0,
% \end{align}
% \end{subequations}
To compute the solution ${\bf w}(t)$ quickly, the numerical algorithms were programmed in CUDA C++ language for parallel execution on general-purpose graphics processing units (GPUs). For ease of use, the CUDA program was then wrapped in a MATLAB mex-function.

The functional gradient $\mathcal{G}(t)$ is computed as
\begin{align}
   \mathcal{G}(t) = \alpha E(t) + {\bf b}^\tp\bm{\lambda}(t),
\end{align}
where $\bm{\lambda}$ is a Lagrange multiplier that obeys the {\it adjoint equations}
\begin{subequations}
  \label{eq:main_ivp_lamb}
  \begin{align}
    \label{eq:main_dot_lamb}
    -\dot{\bm{\lambda}} &= (L+J_F)^\tp \bm{\lambda}, \\
    \label{eq:main_lambT}
    \bm{\lambda}(T) &= R{\bf w}(T).
  \end{align}
\end{subequations}
The ODE system \eqref{eq:main_dot_lamb} has as many variables as the system \eqref{eq:1}. Hence, for efficiency, the numerical solution of \eqref{eq:main_dot_lamb} was programmed in CUDA C++ language and wrapped in a MATLAB mex-function too.


Given a signal $E_s(t)$, the gradient descent method utilizes the functional gradient to produce an improved signal $E_{s+\Delta s}(t)$ using
\begin{align}
  \label{eq:grad_desc}
  E_{s+\Delta s}(t) = E_s(t) - \Delta s\, \mathcal{G}(t)|_{E_s(t)}.
  \end{align}
  More exactly, for small enough $\Delta s$, the new signal $E_{s+\Delta s}(t)$ reduces $\mathcal{L}$,
  \begin{equation*}
    \mathcal{L}[E_{s+\Delta s}(t)] < \mathcal{L}[E_s(t)].
  \end{equation*}
  Therefore, iteration of \eqref{eq:grad_desc} generates a sequence of signals $\{E_s(t), E_{s'}(t), E_{s''}(t),\ldots\}$ that converges to one of many local minima of $\mathcal{L}$. Some of such minima correspond to low-energy defibrillating signals $E(t)$.

The next section describes the installation and operation of the programs that solve Eq. \eqref{eq:1} and perform the gradient descent iteration \eqref{eq:grad_desc}.
  
%The software suite includes the program \code{plot_Et_and_save_fk_forward_Et_mex_scratch.m} that computes the solution ${\bf w}(t)$ of \eqref{eq:1} and saves it to a file. Another program, \code{simple_grad_desc_cuda_phoenix_3.m}, executes the gradient descent iteration \eqref{eq:grad_desc} and stores the sequence of signals $\{E_s(t), E_{s'}(t), E_{s''}(t),\ldots\}$ in files. The installation and operation of these programs are described next.

\section{Software compilation and operation}
The following instructions are to be executed in a Linux operating system. They include Linux commands, identified with the prompt \code{\$}, and MATLAB commands, starting with \code{>>}.
Copy the file \code{adjoint_optimization.tar.gz} to a directory in your machine where you would like to keep the programs. Expand the file with
\begin{lstlisting}
$ tar -xzvf adjoint_optimization.tar.gz
\end{lstlisting}
This creates the directory \code{adjoint_optimization} containing the MATLAB scripts (extension \code{.m}) and the source code of the mex-functions.

\subsection{Forward evolution equation}
The script \code{save_fk_forward_Et.m} computes the solution ${\bf w}(t)$ of the forward evolution equation \eqref{eq:1} and saves it to a file. This program calls the mex-function \code{fk_forward_Et_mex},
which must be generated by compilation of the source files, using
\begin{lstlisting}
>>  mexcuda -output fk_forward_Et_mex fk_forward_Et_param_mex.cpp fk_forward_Et_mono_PARAMETERS.cu
\end{lstlisting}
The function \code{mexcuda} executes  the CUDA compiler \code{nvcc} from either the CUDA toolkit installed with MATLAB or a system-wide CUDA toolkit.

\code{save_fk_forward_Et.m} reads input data from the file \code{infile_forward}. An example of this file in displayed in Listing \ref{code:infile}. It contains MATLAB-style assignments and accepts MATLAB-style comments, starting with \code{\%}. Table \ref{tab:input} explains the meaning of the input variables. The example input files referred to in Listing \ref{code:infile} are provided in directory \code{adjoint_optimization/data}.

%%%%% LSTINPUTLISTING %%%%%%%%%%
\lstinputlisting[firstline=2, caption=Input file \code{infile_forward} for \code{save_fk_forward_Et.m}, label=code:infile]{../src/infile_forward} 

\begin{table}[h]
  \centering
  \begin{tabular}{|l|p{10cm}|}
    \hline
    {\bf Variable} & {\bf Description} \\ \hline
    \code{Et_file} & File with $E(t)$ in ${\rm dV}/{\rm cm}$. The script also reads from this file the final time $T$. If set to the empty string, the script assumes $E(t) = 0$. \\ \hline
    \code{final_time} & Final time $T$  in milliseconds. Ignored, unless \code{Et_file} is the empty string. \\ \hline
    \code{nstep} & Number of time steps at which ${\bf w}(t)$ is saved. \\ \hline
    \code{icondfile} & File with initial tissue state ${\bf w}_0$, ${\bf w}(0) = {\bf w}_0$. \\ \hline
    \code{outfile} & File where ${\bf w}(t)$ will be saved. Set to the empty string to enable the use of \code{outfile_prefix}. \\ \hline
    \code{outfile_prefix} & If \code{outfile} is the empty string, the name of the file where ${\bf w}(t)$ will be saved is composed as the concatenation of  \code{outfile_prefix} and \code{Et_file}.\\ \hline
    \code{mex_function} & mex-function that computes ${\bf w}(t)$.\\ \hline
    \code{grid_file} & File with distribution of non-conducting patches in the tissue.\\ \hline
    \code{in_path} & Directory where input files are read from.\\ \hline
    \code{out_path} & Directory where output files are saved to.\\ \hline
  \end{tabular}
  \caption{Meaning of variables in Listing \ref{code:infile}.}
  \label{tab:input}
\end{table}
The script is executed by simply typing its name (without the extension) at the MATLAB prompt
\begin{lstlisting}
>> save_fk_forward_Et
\end{lstlisting}
which, for the given input variables, creates the output file \code{ys_mex_Et_pulse_N_1_E0_5.00_t0_20.0_300ms.mat} containing the solution ${\bf w}(t)$. This solution is plotted by the script \code{plot_ys.m}, described in Sec. \ref{sec:plotting}.



\subsection{Gradient descent}
The script \code{simple_grad_desc.m} performs the gradient descent iteration \eqref{eq:grad_desc} to generate the sequence of electric field signals $\{E_s(t), E_{s'}(t), E_{s''}(t),\ldots\}$. To compute the functional gradient, the program calls the mex-function \code{fk_grad_Et_mex}, which solves the systems of equations \eqref{eq:1} and \eqref{eq:main_dot_lamb}. This mex-function is generated by compilation of the source files using
\begin{lstlisting}
>> mexcuda -output fk_grad_Et_mex fk_grad_Et_param_mex.cpp  fk_grad_Et_mono_PARAMETERS.cu
\end{lstlisting}

Every time a  batch of certain number of iterations (set through the input variable \code{nmod}) is completed, the program saves to a file the values of $\mathcal{L}$, $\mathcal{N}$, and $\mathcal{M}$ for each iteration in the batch and $E_s(t)$ for the final iteration. The output filename is created as a concatenation of the following: (i) the string \code{simple_grad_desc_step_}, (ii) a string with the date at execution time, (iii) the string in the input variable \code{jobid} (which could be the job ID, if the script is executed in batch mode), (iv) the number of gradient descent iterations so far, and (v) the extension \code{.mat}. For instance \code{simple_grad_desc_step_2024-nov-02-17-32-50_000001_50.mat}.

\code{simple_grad_desc.m} reads input data from the file \code{infile_grad}. An example of this file in displayed in Listing \ref{code:infile_grad}. Table \ref{tab:input_grad} explains the meaning of the input variables.
%%%%% LSTINPUTLISTING %%%%%%%%%%
\lstinputlisting[firstline=2, caption=Example input file \code{infile_grad} for \code{simple_grad_desc.m}, label=code:infile_grad]{../src/infile_grad}


\begin{table}[h]
  \centering
  \begin{tabular}{|l|p{10cm}|}
    \hline
    {\bf Variable} & {\bf Description} \\ \hline
\code{alpha} & Value of parameter $\alpha$ in Eq. \eqref{eq:LNM} in units of ${\rm cm}^2/({\rm dV}^2\,{\rm ms})$\\ \hline
\code{ds} & Value of $\Delta s$ in Eq. \eqref{eq:grad_desc} in units of $({\rm dV}/{\rm cm})^2$\\ \hline
\code{ns} & Number of iterations of the recurrence relation in Eq. \eqref{eq:grad_desc}\\ \hline
\code{final_time} & Final time $T$  in milliseconds.\\ \hline
\code{gamma} & Vector $(\gamma_1, \gamma_2, \gamma_3)$ used in the computation of $\mathcal{M}$ in Eq. \eqref{eq:LNM} \\ \hline
\code{nmod} & Number of iterations of Eq. \eqref{eq:grad_desc} at which $\mathcal{L},\mathcal{N},\mathcal{M}$ and $E_s(t)$ are saved to a file\\ \hline
\code{Et_seed_file} & File with initial electric field signal $E_0(t)$ in ${\rm dV}/{\rm cm}$.\\ \hline
\code{icondfile} & File with initial tissue state ${\bf w}_0$, ${\bf w}(0) = {\bf w}_0$.\\ \hline
\code{grid_file} & File with distribution of non-conducting patches in the tissue. \\ \hline
\code{S_file} & File with matrices used in the computation of $\mathcal{M}$. \\ \hline
\code{mex_function} & mex-function that computes the functional gradient $\mathcal{G}(t)$. \\ \hline
\code{in_path} & Directory where input files are read from.\\ \hline
\code{out_path} & Directory where output files are saved to.\\ \hline
\code{jobid} & String included in the name of the output files. \\ \hline
  \end{tabular}
  \caption{Meaning of the variables in Listing \ref{code:infile_grad}.}
  \label{tab:input_grad}
\end{table}

The script is executed by typing its name at the MATLAB prompt,
\begin{lstlisting}
>> simple_grad_desc
\end{lstlisting}
The script \code{plot_LMN.m}, described in Sec. \ref{sec:plotting}, plots the data of some example output files.

\subsection{Plotting scripts}
\label{sec:plotting}
The following scripts plot the results of \code{save_fk_forward_Et.m} and \code{simple_grad_desc.m}.

\begin{itemize}
\item \code{plot_ys.m}: reads ${\bf w}(t)$ from a file written by \code{save_fk_forward_Et.m} and plots the voltage variable on the two-dimensional domain for several times. The file read by the script must be created by execution of \code{save_fk_forward_Et.m} with the provided input data. %The plot of the initial state is shown in Fig. \ref{}
%   \begin{figure}[h]
%   \centering
%   \includegraphics[width=10cm]{spirals.jpg}
%   \caption{Initial voltage state.}
% \end{figure}


  
\item \code{plot_LMN.m}: reads a group of files written by \code{simple_grad_desc.m} and plots $\mathcal{L}, \mathcal{N}$ and $\mathcal{M}$ as functions of $s$ and the final electric field signal $E_s(t)$. The example files read by the script are provided in directory \code{adjoint_optimization/data}.

\end{itemize}

\section{Test}
This software has been tested with MATLAB Release R2021a using the \code{nvcc} compiler installed with MATLAB, on an NVIDIA Tesla V100 PCIe 16 GB GPU.

% \bibliographystyle{plain}
% \bibliographystyle{abbrv}
\bibliographystyle{plainurl}
\bibliography{references}
%%% -------------------- END DOCUMENT -------------------------
\end{document}

%